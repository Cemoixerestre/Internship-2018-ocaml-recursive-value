\documentclass{article}

\usepackage[utf8]{inputenc}
\usepackage{mathpartir}
\usepackage[nottoc, notlof, notlot]{tocbibind}
\usepackage{hyperref}

\title{}
\author{Alban Reynaud}
\date{}

\begin{document}

\maketitle

\section{Jeremy's Checker}

The current OCaml's recursive-value checker has been written by Jeremy Yallop.
\cite{Yallop}.

\subsection{Modes}

This checker uses three \textit{access modes} to describe the way variables are
accessed in an expression.

These modes are:
\begin{description}
  \item[Deref] : the value of a variable is accessed.
  \item[Guarded] : the address of a variable is either placed in a constructor,
    either in an expression that is lazily evaluated, either unused.
  \item[Unguarded] : the address of a variable is not used in a guarded
    context.
\end{description}

\subsection{Types and Environments}
Access modes are used to describe a type-system.

In this system, the type of a variable $x$ is a map that associate every
variable used in $x$'s definition to its access mode.

An environment is a map that associates variables to a type.

\textit{TODO: describe operations of types and environments (guard, discard,
inspect, ...)}

\subsection{Inference Rules}
This checker can be formalized by inference rules, as pointed out by Gabriel
Scherer \cite{SchererRules}.

\begin{mathpar}
  \infer*[rightstyle=\em, right={where $c$ is a constant.}]
    { }
    {\Gamma \vdash c: \emptyset} 

  \infer*{ }{\Gamma, x: A \vdash x: A}

  \infer*[rightstyle=\em, right={when $x \notin \Gamma$}]
    { }
    {\Gamma \vdash x: \emptyset}
\end{mathpar}

\textit{TODO: complete the rules}

\subsection{The Recursive Check Algorithm}

When an expression of the form
\verb"let rec" $x_1$ \verb"=" $e_1$ \verb"and ... and" $x_n$ = $e_n$
is encountered, an envir,

\section{A new system}

\subsection{Overview}
The checker we propose use a simpler type system. Types are just access modes
rather than maps from variables to modes. Consequently, an environment is a map
that associate to variables a mode.

On the previous system, with a deduction of the form: $\Gamma \vdash expr: A$,
the environment $\Gamma$ is the input and the type $A$ is the output.

On this new system, with a deduction of the form: $\Gamma \vdash expr: m$, the
mode $m$ is the input and the environment $\Gamma$ is the output. The idea is
that $m$ represents the mode in which the expression $e$ will be evaluated, and
the environment $\Gamma$ associates each free variable of $e$ to their use in
$e$.

\subsection{Modes}

As the mode \textbf{Guarded} has different meanings, the mode \textbf{Guarded}
is split into three modes:

\begin{description}
  \item[Guarded]: a variable is \textit{guarded} if its address is placed in a
    constructor or stored in the environment of a closure (\textit{TODO: add an
    example about this subtility, or remove it}). An expression is evaluated in     a guarded context if its value is going to be used in a guarded way.
  \item[Delayed]: an expression is \textit{delayed} if it is lazily evaluated.
    Variables contained in a delayed expression are used in a delayed mode.
  \item[Unused]: a variable that is unused.
\end{description}

\subsection{Operations on modes}

\textit{TODO: describe comparison between modes, mode composition and the rule
let ... in ...}

\subsection{Inference Rules}

\begin{mathpar}
  \infer*[rightstyle=\em, right={where $c$ is a constant}]
         { }
         {\emptyset \vdash c: m}

  \infer*{ }
         {x: m \vdash x: m}

  \infer*{\Gamma_1 \vdash e_1: m[Deref] \\ \Gamma_2 \vdash e_2: m[Deref]}
         {\Gamma_1 + \Gamma_2 \vdash e_1~e_2: m}

  \infer*{\Gamma_1 \vdash e_1: m[Guarded] \\
          ...                             \\
          \Gamma_n \vdash e_n: m[Guarded]}
         {\Gamma_1 + ... + \Gamma_n \vdash K(e_1, ..., e_n): m}

  \infer*{\Gamma, x: m_x \vdash e: m[Delayed]}
         {\Gamma \vdash fun~x \leftarrow e: m}

\end{mathpar}

\bibliographystyle{plain}
\bibliography{sources}

\end{document}
