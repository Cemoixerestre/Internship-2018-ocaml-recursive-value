\documentclass{article}

\usepackage[utf8]{inputenc}
\usepackage{mathpartir}
\usepackage[nottoc, notlof, notlot]{tocbibind}
\usepackage{hyperref}

\title{}
\author{Alban Reynaud}
\date{}

\begin{document}

\maketitle

\section{Jeremy's Checker}

The current OCaml's recursive-value checker has been written by Jeremy Yallop.
\cite{Yallop}.

\subsection{Modes}

This checker uses three \textit{access modes} to describe the way variables are
accessed in an expression.

These modes are:
\begin{description}
  \item[Deref] : the value of a variable is accessed.
  \item[Guarded] : the address of a variable is either placed in a constructor,
    either in an expression that is lazily evaluated, either unused.
  \item[Unguarded] : the address of a variable is not used in a guarded
    context.
\end{description}

\subsection{Types and Environments}
Access modes are used to describe a type-system.

In this system, the type of a variable $x$ is a map that associate every
variable used in $x$'s definition to its access mode.

An environment is a map that associates variables to a type.

\textit{TODO: describe operations of types and environments (guard, discard,
inspect, ...)}

\subsection{Inference Rules}
This checker can be formalized by inference rules, as pointed out by Gabriel
Scherer \cite{SchererRules}.

\begin{mathpar}
  \infer*[rightstyle=\em, right={where $c$ is a constant.}]
    { }
    {\Gamma \vdash c: \emptyset} 

  \infer*{ }{\Gamma, x: A \vdash x: A}

  \infer*[rightstyle=\em, right={when $x \notin \Gamma$}]
    { }
    {\Gamma \vdash x: \emptyset}
\end{mathpar}

\textit{TODO: complete the rules}

\subsection{The Recursive Check Algorithm}

When an expression of the form
\verb"let rec" $x_1$ \verb"=" $e_1$ \verb"and ... and" $x_n$ = $e_n$
is encountered, an envir,

\section{A new system}

\subsection{Overview}
The checker we propose use a simpler type system. Types are just access modes
rather than maps from variables to modes. Consequently, an environment is a map
that associate to variables a mode.

On the previous system, with a deduction of the form: $\Gamma \vdash expr: A$,
the environment $\Gamma$ is the input and the type $A$ is the output.

On this new system, with a deduction of the form: $\Gamma \vdash expr: m$, the
mode $m$ is the input and the environment $\Gamma$ is the output. The idea is
that $m$ represents the mode in which the expression $e$ will be evaluated, and
the environment $\Gamma$ associates each free variable of $e$ to their use in
$e$.

\subsection{Modes}

As the mode \textbf{Guarded} has different meanings, the mode \textbf{Guarded}
is split into three modes:

\begin{description}
  \item[Guarded]: a variable is \textit{guarded} if its address is placed in a
    constructor or stored in the environment of a closure (\textit{TODO: add an
    example about this subtility, or remove it}). An expression is evaluated in     a guarded context if its value is going to be used in a guarded way.
  \item[Delayed]: an expression is \textit{delayed} if it is lazily evaluated.
    Variables contained in a delayed expression are used in a delayed mode.
  \item[Unused]: a variable that is unused.
\end{description}

\subsection{Operations on modes}

\textit{TODO: describe comparison between modes, mode composition and the rule
let ... in ...}

\subsection{Inference Rules}

\begin{mathpar}
  \infer*[rightstyle=\em, right={where $c$ is a constant}]
         { }
         {\emptyset \vdash c: m}

  \infer*{ }
         {x: m \vdash x: m}

  \infer*{\Gamma_1 \vdash e_1: m[Deref] \\ \Gamma_2 \vdash e_2: m[Deref]}
         {\Gamma_1 + \Gamma_2 \vdash e_1~e_2: m}

  \infer*{\Gamma_1 \vdash e_1: m[Guarded] \\
          ...                             \\
          \Gamma_n \vdash e_n: m[Guarded]}
         {\Gamma_1 + ... + \Gamma_n \vdash K(e_1, ..., e_n): m}

  \infer*{\Gamma, x: m_x \vdash e: m[Delayed]}
         {\Gamma \vdash fun~x \rightarrow e: m}
\end{mathpar}

\subsection{The let expressions case}
The basic \verb"let" and \verb"let rec" rules are:

\begin{mathpar}
  \infer*{\Gamma_1 \vdash e_1: m[m_x + Guarded] \\
          \Gamma_2, x: m_x \vdash e2: m}
         {\Gamma_1 + \Gamma_2 \vdash let~x~= e_1~in~e_2: m}

  \infer*{\Gamma_1, x: \_ \vdash e_1: m[m_x + Guarded] \\
          \Gamma_2, x: m_x \vdash e2: m}
         {\Gamma_1 + \Gamma_2 \vdash let~rec~x = e_1~in~e_2: m}
\end{mathpar}

The reason why the mode $m[m_x + Guarded]$ is used is that the expression
\verb"e1" must be evaluated before \verb"e2". Therefore, at the time of the
evaluation, this expression is not delayed. Its mode is at least
\textbf{Guarded}. Then, the mode in which the bound variable \verb"x" is used
influences the mode in which the variable used to compute \verb"e1" are used.
Finally, it must be composed by the surrounding mode $m$.

The generalization for multiple \verb"let" bindings is straightforward:
\begin{displaymath}
  \infer*{\Gamma_1, x_1: \_, x_2: \_ \vdash e_1: m[m_{x_1} + Guarded] \\\\
          \Gamma_2, x_1: \_, x_2: \_ \vdash e_2: m[m_{x_2} + Guarded] \\\\
          \Gamma, x_1: m_{x_1}, x_2: m_{x_2} \vdash e: m}
         {\Gamma_1 + \Gamma_2 + \Gamma \vdash
          let~rec~x_1 = e_1~and~x_2 = e_2~in~e}
\end{displaymath}

\subsection{Patterns and Bindings}
However, this presentation of the \verb"let" rules is partial. Let bindings
using patterns, such as \verb"let (x, y) = e".

There are two kind of patterns to consider:
\begin{description}
  \item[Destructive Patterns]: patterns such as \verb"(x, y)" or \verb"Some x",
    that need an inspection of the matched expression.
  \item[Non-destructive patterns]: patterns such as \verb"x" or \verb"_" that
    need no inspection.
\end{description}

Let $p$ be a pattern, and $mode(p)$ be \textbf{Deref} if the pattern $p$ is
destructive, \textbf{Guarded} otherwise. A rule for \verb"let" expressions with
patterns is:

\begin{displaymath}
  \infer*{\Gamma_1 \vdash p: ms=e_1~in~m \\
          \Gamma_2, vars(p): ms \vdash e_2: m}
         {\Gamma_1 + \Gamma_2 \vdash let~p=e_1~in~e_2: m}
\end{displaymath}

with the new notation:
\begin{displaymath}
  \infer{\Gamma \vdash e: m[\sum ms + mode(p)]}
        {\Gamma \vdash p:ms=e~in~m}
\end{displaymath}

where $ms$ is the list of all the modes of the variables appearing in the
pattern $p$, and $\sum ms$ is the maximal mode of $ms$ (\textbf{Unused} if $ms$
is empty).

\bibliographystyle{plain}
\bibliography{sources}

\end{document}
