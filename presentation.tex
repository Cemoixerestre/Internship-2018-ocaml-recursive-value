\documentclass{beamer}

\usepackage[utf8]{inputenc}
\usepackage{graphicx}
\usepackage{amsmath}
\usepackage{mathpartir}
\usepackage{listings}
\usepackage{minted}
\usepackage[french]{babel}

\usetheme{Warsaw}

\newcommand{\tfun}[2]{\mathtt{fun}~#1 \to #2}
\newcommand{\tapply}[2]{#1~#2}
\newcommand{\tlet}[3]{\mathtt{let~}#1=#2\mathtt{~in~}#3}
\newcommand{\tifthenelse}[3]{\mathtt{if~}#1\mathtt{~then~}#2\mathtt{~else~}#3}

\newcommand{\Deref}{\mathsf{Deref}}
\newcommand{\Unguarded}{\mathsf{Unguarded}}
\newcommand{\Guarded}{\mathsf{Guarded}}
\newcommand{\Delayed}{\mathsf{Delayed}}
\newcommand{\Unused}{\mathsf{Unused}}

\newcommand{\backupbegin}{
   \newcounter{finalframe}
   \setcounter{finalframe}{\value{framenumber}}
}
\newcommand{\backupend}{
   \setcounter{framenumber}{\value{finalframe}}
}

\addtobeamertemplate{navigation symbols}{}{%
  \usebeamerfont{footline}%
  \usebeamercolor[fg]{footline}%
  \hspace{1em}%
  \insertframenumber/\inserttotalframenumber
}

\title{Rethinking OCaml recursive values}
\author{Alban Reynaud}
\institute{Inria Saclay, équipe Parsifal, supersvisé par Gabriel Scherer}
% \titlegraphic{}
\date{4 Septembre 2018}

\begin{document}

\begin{frame}
  \titlepage
\end{frame}

\begin{frame}
  \tableofcontents
\end{frame}

\section{Introduction}

\begin{frame}
  \frametitle{Contexte}
  \begin{itemize}
    \item \mintinline{ocaml}|let rec f x = ... and g y = ...|
    \item \mintinline{ocaml}|let rec x = 1 :: x|
    \item \mintinline{ocaml}|let rec x = 1 + x|
  \end{itemize}
\end{frame}

\begin{frame}
  \frametitle{Motivations}
  \begin{itemize}
    \item Avant OCaml 4.06 : vérification à l'aide de règles \textit{ad hoc}.
    \item Depuis : Système proposé par Jeremy Yallop.
    \item Avantage : ressemble à un système de type
    \item Inconvénient : trop complexe
  \end{itemize}
\end{frame}

\begin{frame}
  \frametitle{Mon stage}
  \begin{itemize}
    \item Exprimer le système de Jeremy sous la forme de règles d'inférence.
    \item Concevoir et implémenter un nouveau système.
    \item Également : correction d'un bug.
  \end{itemize}
\end{frame}

\section{Système de Jeremy Yallop}

\begin{frame}
  \frametitle{Modes d'accès}
  \begin{description}
    \item[Deref] la variable est inspectée.

      Exemple : \mintinline{ocaml}|x| dans \mintinline{ocaml}|x + 1|

    \item<2->[Guarded] l'adresse de la variable est utilisée dans un constructeur.

      Exemple : \mintinline{ocaml}|x| dans \mintinline{ocaml}|Some x| ou \mintinline{ocaml}|[x]|

    \item<3->[Delayed] la variable est utilisée dans une expression qui sera évaluée
      plus tard.

      Exemple : \mintinline{ocaml}|x| dans \mintinline{ocaml}|fun () -> x|
  \end{description}
\end{frame}

\begin{frame}
  \frametitle{Ce qui devrait être autorisé}
  \begin{itemize}
    \item Les variables devraient être utilisées en mode $\Guarded$ ou
      $\Delayed$
    \item \mintinline{ocaml}|let rec f x = ... and g y = ...|
    \item \mintinline{ocaml}|let rec x = 1 :: x|
    \item \mintinline{ocaml}|let rec x = 1 + x|
  \end{itemize}
\end{frame}

\begin{frame}
  \frametitle{Système de Jeremy}
  \begin{itemize}
    \item types : $u ::= (x : m)*$
    \item environnements : $\Gamma ::= (x: u)*$
    \item déductions : $\Gamma \vdash e: u$
  \end{itemize}
\end{frame}

\begin{frame}
  \frametitle{Relations entre les modes}
  \begin{itemize}
    \item $\Deref > \Guarded > \Unguarded$
    \item $m + m'$ : mode le plus contraignant
    \item $u + u' : x \rightarrow u(x) + u'(x)$
    \item $\Gamma + \Gamma' : x \rightarrow \Gamma(x) + \Gamma'(x)$
  \end{itemize}
\end{frame}

\begin{frame}
  \frametitle{Algorithme de vérification}
  \begin{itemize}
    \item<1-> \mintinline{ocaml}|let rec|~$x_1 = e_1~$~\mintinline{ocaml}|and ... and|~$x_n = e_n$
    \item<2-> $\Gamma = (x_i: (x_i: \Guarded))*$
    \item<3-> $\Gamma \vdash e_1: u_1$, ..., $\Gamma \vdash e_n: u_n$
  \end{itemize}
\end{frame}

\begin{frame}
  \frametitle{Exemple de règle}
  \begin{mathpar}
  \infer*
    {\Gamma \vdash e_1: T_1 \\
     \Gamma, x : T_1 \vdash e_2: T_2}
    {\Gamma \vdash \tlet{x}{e_1}{e_2}: T_1 + T_2}
  \end{mathpar}
\end{frame}

\section{Nouveau système}

\begin{frame}
  \frametitle{Présentation}
  \begin{itemize}
    \item type = mode
    \item environnement : $\Gamma \vdash (x : m)*$
    \item déductions : $\Gamma \vdash x : m$
  \end{itemize}
\end{frame}

% TODO: réécrire ces deux sections
\begin{frame}
  \frametitle{Composition de modes}
  \begin{itemize}

    \item<1-> \mintinline{ocaml}|fun () -> x + 1|
      \begin{itemize}
        \item<2-> \mintinline{ocaml}|x| est utilisé en mode $\Deref$ dans
          \mintinline{ocaml}|x + 1|
        \item<3-> \mintinline{ocaml}|x + 1| est utilisé en mode $\Delayed$
        \item<4-> \mintinline{ocaml}|x| est utilisé en mode $\Delayed$
      \end{itemize}

    \item<5-> \mintinline{ocaml}|Some(x + 1)|
      \begin{itemize}
        \item<6-> \mintinline{ocaml}|x| est utilisé en mode $\Deref$ dans
          \mintinline{ocaml}|x + 1|
        \item<7-> \mintinline{ocaml}|x + 1| est utilisé en mode $\Guarded$
        \item<8-> \mintinline{ocaml}|x| est utilisé en mode $\Deref$
      \end{itemize}

  \end{itemize}
\end{frame}

\begin{frame}
  \frametitle{Composition de modes}

  \begin{itemize}
    \item $e[e'[x]]$ : $x$ est utilisée en mode $m[m']$
    \item Calcul :
  \end{itemize}

  \begin{displaymath}
    \begin{array}{lll}
      \Deref [m] & = & \Deref \\
      \Delayed [m] & = & \Delayed \\
      \Guarded [m] & = & m \\
    \end{array}
  \end{displaymath}
\end{frame}

\begin{frame}
  \frametitle{Algorithme de vérification}
  \begin{itemize}
    \item<1-> \mintinline{ocaml}|let rec|~$x_1 = e_1~$~\mintinline{ocaml}|and ... and|~$x_n = e_n$
    \item<2-> $\Gamma_1 \vdash e_1: \Guarded$, ...,
              $\Gamma_n \vdash e_n: \Guarded$
  \end{itemize}
\end{frame}

\begin{frame}
  \frametitle{Exemple de règle}

  \begin{mathpar}
  \infer*{\Gamma_1 \vdash e_1: m' \\
          \Gamma_2, x: m_x \vdash e_2: m}
         {\Gamma_1 + \Gamma_2 \vdash \tlet{x}{e_1}{e_2}: m}
  \end{mathpar}

  \begin{itemize}
    \item<2-> $m' \geq m[\Guarded]$
    \item<3-> $m' \geq m[m_x]$
    \item<4-> $m' = m[m_x + \Guarded]$
  \end{itemize}
\end{frame}

\begin{frame}
  \frametitle{Exemple de règle}

  \begin{mathpar}
  \infer*{\Gamma_1 \vdash e_1: m[m_x + \Guarded] \\
          \Gamma_2, x: m_x \vdash e_2: m}
         {\Gamma_1 + \Gamma_2 \vdash \tlet{x}{e_1}{e_2}: m}
  \end{mathpar}
\end{frame}

\begin{frame}[fragile]
\frametitle{Implémentation}
  \begin{mathpar}
    \infer*
      {\Gamma_c \vdash e_c: m[\Deref] \\
       \Gamma_1 \vdash e_1: m       \\
       \Gamma_2 \vdash e_2: m}
      {\Gamma_c + \Gamma_1 + \Gamma_1 \vdash \tifthenelse{e_c}{e_1}{e_2}: m}
  \end{mathpar}

  \pause

  \begin{minted}[fontsize=\scriptsize]{ocaml}
let rec expression : mode -> Typedtree.expression -> Env.t =
  fun mode exp -> match exp.exp_desc with
    (* ... *)
    | Texp_ifthenelse (cond, ifso, ifnot) ->
      let env_cond = expression (compose mode Dereferenced)
                                cond in
      let env_ifso = expression mode ifso in
      let env_ifnot = option expression mode ifnot in
      Env.join env_cond (Env.join env_ifso env_ifnot)
  \end{minted}
\end{frame}

\begin{frame}
  \frametitle{Implémentation}
  \begin{itemize}
    \item Passe tous les tests
    \item Une pull request a été soumise
  \end{itemize}
\end{frame}
  
\section*{Conclusion}

\begin{frame}
  \frametitle{Conclusion}
  Pendant ce stage :
  \begin{itemize}
    \item j'ai relu et exprimé le système de Jeremy Yallop sous forme de règles
      d'inférences
    \item j'ai conçu et implémenté un nouveau système
    \item j'ai également corrigé un bug du compilateur
  \end{itemize}
\end{frame}

\begin{frame}
  \frametitle{Perspectives}
  \begin{itemize}
    \item Possibilité de faire une preuve de la correction du système
    \item Étendre le système
    \item Utiliser les données pour la compilation
  \end{itemize}
\end{frame}

\appendix
\backupbegin

\begin{frame}
  \frametitle{Autres modes d'accès}
  \begin{itemize}
    \item Problème : que faire de la déclaration \mintinline{ocaml}|let rec x = x| ?
    \item<2-> Mode $\Unguarded$ ($\Deref > \Unguarded > \Guarded$)
    \item<3-> Également un mode $\Unused$ dans le nouveau système
    \item<4-> Dans le système de Jeremy, les modes $\Guarded$, $\Delayed$ et
      $\Unused$ sont confondus.
  \end{itemize}
\end{frame}

\begin{frame}[fragile]
  \frametitle{Correction d'un bug}
  \begin{itemize}
    \item<1-> Possibilité de définir de classes :
      \begin{minted}[mathescape=true]{ocaml}
class c = object ... end
class c = let x = e in ... [classexpr]
      \end{minted}
    \item<2-> Les expressions \mintinline{ocaml}|e| et
      \mintinline{ocaml}|[classexpr]| étaient typées séparément.
    \item<3-> 
      \begin{minted}{ocaml}
class c = let x () = new c in
          let y = x () in
          object end
      \end{minted}
  \end{itemize}
\end{frame}

\backupend

\end{document}
