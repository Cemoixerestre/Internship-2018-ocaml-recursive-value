\documentclass{article}

\usepackage[utf8]{inputenc}
\usepackage{listings}
\usepackage{url}
\usepackage{hyperref}
\usepackage{amsmath}
\usepackage{mathpartir}
% \usepackage[french]{babel}

\lstset{language=caml}

% Some macros:
\newcommand{\tfun}[2]{\mathtt{fun}~#1 \to #2}
\newcommand{\tapply}[2]{#1~#2}

\newcommand{\Deref}{\mathsf{Deref}}
\newcommand{\Unguarded}{\mathsf{Unguarded}}
\newcommand{\Guarded}{\mathsf{Guarded}}
\newcommand{\Delayed}{\mathsf{Delayed}}
\newcommand{\Unused}{\mathsf{Unused}}

\newcommand{\inspect}{\mathsf{inspect}}
\newcommand{\guard}{\mathsf{guard}}
\newcommand{\discard}{\mathsf{discard}}
\newcommand{\delay}{\mathsf{delay}}
% \newcommand{\mode}[1]{\text{\textbf{#1}}}

\title{Rethinking OCaml Recursive Values}
% Should I include a kind of subtitle to indicate that this is a "rapport de
% stage" for my licence?
\author{Alban Reynaud}
\date{}

\begin{document}

\maketitle

This work has been done in an internship in Inria Saclay, team Parsifal. It was
supervised by Gabriel Scherer.

\section{Introduction}
OCaml is a language of the ML family. One of the features of the language is the
\lstinline|let rec| operators. It is usually used to define recursive
functions, but it can also be used to define recursive values.

For example, the following code: \lstinline|rec x = 1 :: x| creates an infinite
list containing only ones.

However, not all definitions can be computed. For example, definitions like:
\lstinline|let rec x = x + 1| should be rejected by the compiler.

Previously, this check relied on \textit{ad hoc} rules \cite{PreviousRules},
which caused several issues. In 2016, Jeremy Yallop proposed a new check,
designed as a simplified typing system \cite{Yallop}. One of the problem is that
his system is too complex, and it is hard to reason about.

During this internship, I investigated the current system and (following the
intuition of Gabriel Scherer) and rewrote it as an inference rules system. Then,
I proposed a simpler system. I implemented it in the compiler and successfully
managed to pass the testsuite. Also, examining Jeremy Yallop's code, Gabriel and
I found a bug and sent a pull request.

\section{The current system}

\subsection{Access modes}
Three \textit{access modes} are used in this system. These modes are:
\begin{description}
  \item[Deref] : the value of a variable is accessed.
  \item[Guarded] : the address of a variable is either placed in a constructor,
    either in an expression that is lazily evaluated, either unused.
  \item[Unguarded] : the address of a variable is not used in a guarded
    context.
\end{description}

For example:
\begin{itemize}
  \item In the expression \lstinline|x + 1|, \lstinline|x| is accessed in the
    $\Deref$ mode because its value must be accessed before being used as
    a function argument.
  \item In the expression \lstinline|1 :: x|, \lstinline|x| is accessed in the
    $\Guarded$ mode. The address of \lstinline|x| is used in the
    construction of the list, be it does not require the evaluation of
    \lstinline|x|.
  \item In the expression \lstinline|fun () -> x|, \lstinline|x| is accessed in
    the $\Guarded$ mode, because the body of this function is lazily
    evaluated, so it does not require the evaluation of \lstinline|x|.
  \item Considere the expression \lstinline |let rec x = let y = x in ...|
    In the definition of \lstinline|y|, \lstinline|x| is neither evaluater, nor
    placed in a constructor, nor lazily evaluated. \lstinline|x| is considered
    being used in the $\Unguarded$ mode.
\end{itemize}

\subsection{Types and Environments}
In this system, types are maps that associate variables an access mode.
Intuitively, the type of an expression give the use of variables in the
definition of the expression.

An environment is a map that associate variables to a type.

\subsection{Operations}
Some operations are defined on modes:
\begin{displaymath}
  \begin{array}{lll}
    \guard(\Unguarded)              & = & \Guarded \\
    \guard(m \neq \Unguarded)       & = & m        \\
    \inspect(m)                     & = & \Deref   \\
    \delay(m)                       & = & \Guarded \\
    \discard(m)                     & = & \guard(m)
  \end{array}
\end{displaymath}

These operations are generalized on types.
% Should I describe more precesily what I mean, or is it clear?

It is possible to define an total order on modes:
$$\Deref > \Unguarded > \Guarded$$
We have $m > m'$ if the mode $m$ is more constraining than the mode $m'$.

So, let $m + m'$ denote the maximal mode between $m$ and $m'$, with respect to
this order relation. Note that this operation is called \textit{prec} in
Jeremy Yallop's code.

We can generalize this definition for types and environments:
\begin{displaymath}
  \begin{array}{lll}
    (t + t')           & : x \rightarrow & t(x) + t'(x) \\
    (\Gamma + \Gamma') & : x \rightarrow & \Gamma(x) + \Gamma'(x)
  \end{array}
\end{displaymath}

with:

\begin{displaymath}
  \begin{array}{lll}
    t(x) = \Guarded       & \text{if} & x \notin dom(t) \\
    \Gamma(x) = \emptyset & \text{if} & x \notin dom(\Gamma)
  \end{array}
\end{displaymath}

\subsection{Inference Rules}
Originally, Jeremy Yallop just gave an implementation of this check. Gabriel
Scherer had the intuition that this system could be expressed with inference
rules \cite{GascheComment1, GascheComment2}.

This system works as follow: conclusions have the form
$$\Gamma \vdash e: t$$
where $\Gamma$ is an environment, $e$ is an expression and $t$ a type. The
environment is the input and the type is the ouput.

The main rules are:
\begin{mathpar}
  \infer*
    {c~\text{is a constant}}
    {\Gamma \vdash c: \emptyset}

  \\

  \infer*{ }
         {\Gamma, x: t \vdash x: t}

  \infer*%[rightstyle=\em, right={when $x \notin \Gamma$}]
    {x \notin \Gamma}
    {\Gamma \vdash x: \emptyset}

  \\

  \infer*{\Gamma \vdash e_1: t_1 \\
          \Gamma \vdash e_2: t_2}
         {\Gamma \vdash \tapply{e_1}{e_2}: \inspect(t_1 + t_2)}

  \infer*{\Gamma \vdash e: t}
         {\Gamma \vdash \tfun x e: \delay(t)}

  \\

  \infer*
    {(\Gamma \vdash e_i: t_i)^i}
    {\Gamma \vdash K(e_1, ..., e_n): \guard(t_1 + ... + t_n)}
\end{mathpar}
% Should I give a short explanation?

\subsection{The Check Algorithm}
The check is performed recursively. Every node of the abstract syntax tree
\footnote{defined in the module \texttt{Typedtree}, that is defined is the
files \texttt{typing/typedtree.mli} and \texttt{typing/typedtree.ml}} is
checked. If it is a node with the form:
\begin{lstlisting}[mathescape=true]
  let rec $x_1$ = $e_1$ and ... and $x_n$ = $e_n$
\end{lstlisting}
An environment $\Gamma = (x_i: (x_i: \Unguarded))^i$ is created.
Then, the types of the $e_i$ expressions are computed in the $\Gamma$
environment. If a variable is used in mode $\Unguarded$ or $\Deref$ in one of
these types, then the compiler reject this expression.

\subsection{Examples}
\textit{TODO}

\subsection{Bug in the implementation}
When examining the code, Gabriel and I found a bug. It appears in the recursive
class declarations check.

Indeed, it is possible to define classes with the syntax:
\begin{lstlisting}[mathescape=true]
class $a$ = let rec $x_1$ = $e_1$ and ... and $x_n$ = $e_n$ in $e$
\end{lstlisting}
The compiler check that the \lstinline|let rec| bindings are correct. In
particular, it is not allowed to use the recursively defined class $a$ in an
unguarded mode.

%TODO: give more details. Show the rules? The code?
The problem was that in the previous check, the expressions affected to
variables bound in the \lstinline|let rec| and the class expression $e$ were
typed separately. For example, in the definition:
\begin{lstlisting}[mathescape=true]
class a = let rec x = $e_x$ in let y = $e_y$ in $e$
\end{lstlisting}
it is possible to make an incorrect use of the variable $x$ in the expression
$e_y$.

Gabriel found that the definition:
\begin{lstlisting}
class a = let rec x = new a in let _y = x () in object end
\end{lstlisting}
caused a segfault.

I wrote a patch that fix this bug, added new test and sent a pull request,
with the help of Gabriel \cite{PullRequest}.

% Note that this bug only appears since the version 4.06.

\bibliographystyle{plain}
\bibliography{sources}

\end{document}
